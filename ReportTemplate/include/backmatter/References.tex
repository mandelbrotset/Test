% CREATED BY DAVID FRISK, 2015
\begin{thebibliography}{69}
\bibitem{Reference} Y. Dubinsky, J. Rubin, T. Berger, S. Duszynski, M. Becker, K. Krzysztof, “An Exploratory Study of Cloning in Industrial Software Product Lines”, 2013

\bibitem{Reference} V. Driessen. “A successful Git branching model” Internet: http://nvie.com/posts/a-successful-git-branching-model/, Jan. 5, 2010 [Jan. 15, 2016].

\bibitem{Reference} GitHub Inc. “Understanding the GitHub Flow”, Internet: https://guides.github.com/introduction/flow/, Dec. 12, 2013 [Jan. 15, 2016]

\bibitem{Reference} Atlassian, “Using Branches”, Internet: https://www.atlassian.com/git/tutorials/using-branches/git-checkout, [Jan. 15, 2016]

\bibitem{Reference} GitHub Inc. “Using pull-requests” Internet: https://help.github.com/articles/using-pull-requests/, [Jan. 15, 2016]

\bibitem{Reference} M. Antkiewicz, W. Ji, T. Berger, K. Czarnecki, T. Schmorleiz, R. Lämmel, S. Stanciulescu, A. Wasowski, I. Schaefer, “Flexible Product Line Engineering with a Virtual Platform”, 2014

\bibitem{Reference} S. Apel, J. Liebig, C. Lengauer, C. Kästner, W. Cook, “Semistructured Merge in Revision Control Systems”, 2010

\bibitem{Reference} T. Mens. “A State-of-the-Art Survey on Software Merging” IEEE TRANSACTIONS ON SOFTWARE ENGINEERING,VOL. 28,NO. 5, May 2002, pp, 449-462.
http://uff-labgc-2010-2-grupo5.googlecode.com/svn/trunk/seminarios/artigos/mens2002.pdf

\bibitem{Reference} S. Apel, O. Leßenich, C. Lengauer, “Structured Merge with Auto-Tuning: Balancing Precision and Performance”, 2012

\bibitem{Reference} G. Cavalcanti, P. Accioly, P. Borba, “Assessing Semistructured Merge in Version Control Systems: A Replicated Experiment”, 2015

\bibitem{Reference} P. Accioly, “Understanding Conflicts Arising from Collaborative Development”, 2015

\bibitem{Reference} GitHub Inc. “GitHub Developer”, Internet: https://developer.github.com/v3/\#rate-limiting, [Jan. 15, 2016]
\end{thebibliography}
