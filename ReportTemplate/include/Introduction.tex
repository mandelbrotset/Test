% CREATED BY DAVID FRISK, 2015
\chapter{Introduction}
\setlength{\parindent}{0pt}
When developing software products using a version-control system, branching is often used, both to support variability [1] and to develop new features. It is of vital importance that the merging of these branches works smoothly, even when a conflict occurs.

Resolving merge conflicts, such as those arising from changes to different variants of features, is difficult. Merging might require refactoring the class hierarchy, introducing design patterns, or adding parameters to the feature. If it would be possible to develop a tool that can provide automated conflict resolution in this case, it would be of great value, since resolving such conflicts is a recurring problem that is solved manually today. The problem is that the development of such a tool requires more understanding of how the merge-conflict resolution is performed. For having representative results, one would need to study the resolutions in large codebases, such as from GitHub/BitBucket. This requires some automated analysis.

The goal of this work is to conduct a feasibility study that aims at investigating whether analysis of merge-conflict resolution patterns in large codebases can be automated. We focus on open-source projects, such as those being hosted on GitHub or BitBucket.

Understanding which patterns are used for resolving merge conflicts allows for future development of an autonomous merge-conflict tool which is our long-term goal. Towards this end, this study aims at answering the following research questions:
\begin{itemize}
\item RQ1. Which patterns exists for resolving merge-conflicts?
\item RQ2. How frequent are these patterns in large code-bases?
\item RQ3. Is it feasible to statically analyze conflict-resolution patterns in real-world, large version histories of open-source projects, and how?
\item RQ4. What kind of mining and analysis infrastructure is needed for such a study?
\end{itemize}

Our main working hypothesis is that it is in fact feasible to automatically recognize and analyze patterns from all the metadata available about projects and from statically analyzing code. By studying examples of popular projects with many branches or forks, by developing a mining infrastructure, and by investigating to what extent static code-analysis tools can be utilized for recognizing any patterns, we will work towards testing this hypothesis.




